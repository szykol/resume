%!TEX TS-program = xelatex
%!TEX encoding = UTF-8 Unicode
% Awesome CV LaTeX Template
%
% This template has been downloaded from:
% https://github.com/posquit0/Awesome-CV
%
% Author:
% Claud D. Park <posquit0.bj@gmail.com>
% http://www.posquit0.com
%
% Template license:
% CC BY-SA 4.0 (https://creativecommons.org/licenses/by-sa/4.0/)
%


%%%%%%%%%%%%%%%%%%%%%%%%%%%%%%%%%%%%%%
%     Configuration
%%%%%%%%%%%%%%%%%%%%%%%%%%%%%%%%%%%%%%
%%% Themes: Awesome-CV
\documentclass[]{awesome-cv}
\usepackage{textcomp}
%%% Override a directory location for fonts(default: 'fonts/')
\fontdir[fonts/]

%%% Configure a directory location for sections
\newcommand*{\sectiondir}{resume/}

%%% Override color
% Awesome Colors: awesome-emerald, awesome-skyblue, awesome-red, awesome-pink, awesome-orange
%                 awesome-nephritis, awesome-concrete, awesome-darknight
%% Color for highlight
% Define your custom color if you don't like awesome colors
\colorlet{awesome}{awesome-red}
%\definecolor{awesome}{HTML}{CA63A8}
%% Colors for text
%\definecolor{darktext}{HTML}{414141}
%\definecolor{text}{HTML}{414141}
%\definecolor{graytext}{HTML}{414141}
%\definecolor{lighttext}{HTML}{414141}

%%% Override a separator for social informations in header(default: ' | ')
%\headersocialsep[\quad\textbar\quad]
    \begin{document}
    
%%%%%%%%%%%%%%%%%%%%%%%%%%%%%%%%%%%%%%
%     Profile
%%%%%%%%%%%%%%%%%%%%%%%%%%%%%%%%%%%%%%
\begin{center}
	\headerfirstnamestyle{Szymon} \headerlastnamestyle{Kołton} \\
	\vspace{2mm}
	{\faEnvelope\ szymonkolton@gmail.com} | {\faMobile\ +48*********} | {\faMapMarker\ Kraków} | {September 1998}
\end{center}
%%%%%%%%%%%%%%%%%%%%%%%%%%%%%%%%%%%%%%
%     Experience
%%%%%%%%%%%%%%%%%%%%%%%%%%%%%%%%%%%%%%
\cvsection{Work Experience}
\begin{cventries}

	\cventry
	{Software Developer - Go \& Python}
	{Falcon V Systems | \descriptionstyle{Telco}}
	{Gdynia - Remote}
	{Oct 2022 – Present}
	{\begin{cvitems}
		\item {Contributed to the development of advanced software solutions for both Remote PHY and Remote MAC-PHY architectures at a leading telecommunications company specializing in Open DAA (Distributed Access Architecture).}
		\item {Developed critical components using Go, enhancing the performance and scalability of the products.}
		\item {Played a significant role in creating and managing two key products for Remote PHY networks: a suite of tools for the operation and management of Remote PHY Devices (RPDs).}
		\item {Developed a product that complied with the CableLabs FMA MAC Manager Interface Specification, designed specifically for managing Remote MAC-PHY Devices (RMDs).}
		\item {Operated within an Agile framework, delivering incremental updates every two weeks and engaging in regular stakeholder demos to incorporate feedback and refine the products.}
		\item {Ensured products compatibility with CableLabs specifications to meet industry standards.}
		\item {Utilized GitLab CI to bolster product stability and ensure reliable delivery by automating the execution of race tests, unit tests, and integration tests. This approach effectively prevented regressions and confirmed that all features functioned as expected, facilitating smoother and more dependable releases.}
		\item {Utilized Kubernetes with Helm for deployment orchestration and leveraged Docker Compose to run integration tests, facilitating a consistent testing environment and ensuring reliable deployment pipelines.}
		\end{cvitems}}
	\cventry
	{Software Engineer - Python}
	{Alvaria (formerly Noble Systems Corporation) | \descriptionstyle{Contact Center Solutions}}
	{Kraków}
	{Nov 2021 – Oct 2022}
	{\begin{cvitems}
		\item {Contributed to the maintenance and continuous improvement of Speech Analytics services. Implementing updates and API enhancements. Collaborated with cross-functional teams to port key APIs from an external gateway into our product, resulting in streamlined code for the API gateway and better alignment with the product's original design.}
		\item {Collaborated within a team to refactor a monolithic on-premise application into microservices, enabling deployment across multiple cloud platforms. The new architecture allowed for platform-specific deployment during service rollout, enhancing flexibility and scalability across AWS, Azure, and Google Cloud.}
		\end{cvitems}}
	\cventry
	{Junior Software Developer - Python}
	{Noble Systems Corporation | \descriptionstyle{Contact Center Solutions}}
	{Kraków}
	{Aug 2019 – Nov 2021}
	{\begin{cvitems}
		\item {Developed and maintained backend products utilizing a third-party speech recognition engine to automate the analysis of real-time calls and post-call recordings. Implemented customer-driven features, including enhancements and bug fixes. Supported up to two versions prior to the current release and collaborated closely with the QA team to ensure all functionality met client expectations.}
		\item {Facilitated the delivery of software products for CentOS 7 systems by creating and distributing RPM packages. Utilized a Jenkins pipeline to automate the packaging process.}
		\item {Contributed to the migration of legacy products from Python 2 and Twisted to Python 3 and asyncio, addressing the challenge of integrating asyncio with existing Twisted-based code. Utilized Twisted’s support for asyncio, along with tools such as six and 2to3, to streamline the transition. The migration resulted in a clearer and more maintainable codebase, significantly simplifying ongoing maintenance efforts.}
		\end{cvitems}}
\end{cventries}
%%%%%%%%%%%%%%%%%%%%%%%%%%%%%%%%%%%%%%
%     Education
%%%%%%%%%%%%%%%%%%%%%%%%%%%%%%%%%%%%%%
\cvsection{Education}
\begin{cventries}
	\cventry
	{BSc in Computer Science}
	{Pedagogical University of Kraków}
	{Kraków}
	{Aug 2017 – Mar 2021}
	{}
\end{cventries}

\vspace{-2mm}
\cvsection{Skills}
\begin{cventries}
	\cventry
	{}
	{\def\arraystretch{1.15}{\begin{tabular}{ l l }
		Go: & {\skill{gomock, zap, dlv, freeconf, kafka-go, mongo-db-driver, msgpack, go-redis, go race}} \\
		Python:  & {\skill{ asyncio, fastapi, psycopg, asyncpg, aiohttp, flask, twisted, pytest }} \\
		Data Stores \& Messaging:  & {\skill{ postgresql, redis, mongodb, kafka }} \\
		VCS: & {\skill { git, github, gitlab, bitbucket, tfs }} \\
		Container Orchestration: & {\skill {kubernetes, helm, docker, docker-compose, docker-hub, k3s, harbor, k9s}} \\
		CI/CD: & {\skill {gitlab CI/CD, github actions, Jenkins}} \\
		Other: & {\skill {azure, jira, restconf, yang, gcp, rcp, scrum, hexagonal architecture, swagger, driver license}} \\
		\end{tabular}}}
	{}
	{}
	{}
\end{cventries}

\vspace{-7mm}

\pagebreak
\cvsection{Languages}
\begin{cventries}
	\cventry
	{}
	{\def\arraystretch{1.25}{\begin{tabular}{ l l }
		English:  & {\skill{ B2}} \\
		Polish:  & {\skill{ Native}} \\
		\end{tabular}}}
	{}
	{}
	{}
\end{cventries}
\vspace{-7mm}

\cvsection{Links}
\begin{cventries}
	\cventry
	{}
	{\def\arraystretch{1.25}{\begin{tabular}{ l l }
		\href{https://www.github.com/szykol}{\faGithubSquare\ \@www.github.com/szykol} \\
		\href{https://www.linkedin.com/in/szymon-kolton}{\faLinkedinSquare\ \@www.linkedin.com/in/szymon-kolton} \\
		\end{tabular}}}
	{}
	{}
	{}
\end{cventries}



\vspace{-5mm}
\emptycvsection{nullop}
\begin{cventries}
	\cventry
	{}
	{}
	{}
	{}
	{I hereby give consent for my personal data included in my application to be processed for the purposes of the recruitment process under the Regulation (EU) 2016/679 of the European Parliament and of the Council of 27 April 2016 on the protection of natural persons with regard to the processing of personal data and on the free movement of such data, and repealing Directive 95/46/EC
	(General Data Protection Regulation).}
\end{cventries}
\end{document}
